\documentclass[compress]{beamer}

\mode<presentation>
\usetheme{Madrid}
\definecolor{Columbia}{RGB}{185,217,235}
\definecolor{Columbia2}{RGB}{0,51,160}
\definecolor{Columbia3}{RGB}{0,114,206}
\setbeamercolor{title}{fg=Columbia2}
\setbeamercolor{frametitle}{fg=Columbia2}
\setbeamercolor{block title}{bg=Columbia, fg=Columbia2}
%\setbeamercolor{block body}{fg=Columbia2}
\setbeamercolor{structure}{fg=Columbia}
\setbeamercolor{item projected}{fg=white}
\setbeamercolor{item}{fg=Columbia2}
\setbeamercolor{subitem}{fg=Columbia2}
\setbeamercolor{section in toc}{fg=Columbia2}
\setbeamercolor{description item}{fg=Columbia}
\setbeamercolor{caption name}{fg=Columbia2}
\setbeamercolor{button}{fg=Columbia2}
\usepackage{graphics}
\usepackage{geometry}
\usepackage{booktabs}
\usepackage{tikz}
\usepackage{amsmath}
\usepackage{bbm}
\usetikzlibrary{decorations.pathreplacing}
\usepackage{multirow, makecell}
\usepackage{float}
\usepackage{fancyvrb}
\usepackage{caption}
\usepackage{subcaption}
\usepackage{adjustbox}
\usepackage{threeparttable}
\usepackage{hyperref}
\usepackage[scaled=0.92]{helvet}
\newenvironment{wideitemize}{\itemize\addtolength{\itemsep}{10pt}}{\enditemize}
\newenvironment{wideenumerate}{\enumerate\addtolength{\itemsep}{10pt}}{\endenumerate}
%\hypersetup{
%colorlinks=true,
%linkcolor=black,
%filecolor=green, 
%urlcolor=blue,
%}
\beamertemplatenavigationsymbolsempty
\setbeamercolor{author in head/foot}{bg=Columbia2, fg=white}
\setbeamercolor{title in head/foot}{bg=Columbia3, fg=white}
\setbeamercolor{date in head/foot}{bg=Columbia, fg=Columbia2}
\setbeamercolor{section in head/foot}{bg=Columbia, fg=Columbia2}
\setbeamercolor{headline}{bg=Columbia}
\setbeamertemplate{footline}{
    \leavevmode%
    \hbox{%
        \begin{beamercolorbox}[wd=.333333\paperwidth,ht=2.25ex,dp=1ex,center]{author in head/foot}%
            \usebeamerfont{author in head/foot}\insertshortauthor
        \end{beamercolorbox}%
        \begin{beamercolorbox}[wd=.333333\paperwidth,ht=2.25ex,dp=1ex,center]{title in head/foot}%
            \usebeamerfont{title in head/foot}\insertshorttitle
        \end{beamercolorbox}%
        \begin{beamercolorbox}[wd=.333333\paperwidth,ht=2.25ex,dp=1ex,right]{date in head/foot}%
            \usebeamerfont{date in head/foot}\insertshortdate{}\hspace*{2em}
            \insertframenumber{} / \inserttotalframenumber\hspace*{2ex} 
        \end{beamercolorbox}}%
        \vskip0pt%
    }
\setbeamercolor{page number in head/foot}{fg=black}
\setbeamertemplate{section in toc}[sections numbered]
\setbeamertemplate{subsection in toc}{\leavevmode\leftskip=3em\rlap{\hskip-1.75em\inserttocsectionnumber.\inserttocsubsectionnumber}\inserttocsubsection\par}
\setbeamerfont{subsection in toc}{size=\footnotesize}
%\setbeamertemplate{headline}{%
  %\begin{beamercolorbox}[ht=5.5ex]{section in head/foot}
    %\vskip2pt\insertnavigation{0.33\paperwidth}\vskip2pt
  %\end{beamercolorbox}%
%}


\makeatletter
\let\@@magyar@captionfix\relax
\makeatother


\title[Recitation 10]{Recitation 10} % Change this regularly
\author[Seung-hun Lee]{Seung-hun Lee}
\institute[Columbia University]{Columbia University}

\date[]{}

\begin{document}
\begin{frame}
\titlepage
\end{frame}

%%%%%%%%%%%%% Section 1. 
\begin{frame}
\frametitle{Note on the final project}
\begin{itemize}
\item Approach the TA's for advice: regression methods, data, or ideas
\item Also, refer to my recitation notes for Diff-in-Diffs and Regression discontinuity methods - I explain how you can use it in a natural experiment context
\item Data: Household surveys are the safe place to start but country level data could work depending on your idea
\item But first, \textbf{be clear about what kind of question you want to explore}: are you looking for a correlation/causal relation? Are you trying to predict something?
\item Who to look for (topic-wise) - all three of us cover basic grounds in most topics, but for specialized stuff...
\begin{itemize}
\item Me: Development/Public/Labor econ (applied microeconomics in general). If you are trying to study health-care markets in the US, I am the wrong person to ask
\end{itemize}
\end{itemize}
\end{frame}


% Capture in one slide
% Separate slide for questions
\begin{frame}
\frametitle{Experiments in Economics}
Motivation
\begin{itemize}
\item You categorize some individuals under treatment group and controlled group and compare the differences across groups and times. 
\item You can do this in lab setting (randomized control trials) or find an exogenous event (natural experiments)
\item They provide a conceptual benchmark for assessing observational studies and can solve many validity threats in regular regressions.
\item Do note that they have a validity threat of their own
\end{itemize}
\end{frame}

\begin{frame}
\frametitle{Potential Outcome Framework}
Setup
\begin{itemize}
\item Start by assuming that the treatment effect is identical for everyone (\textbf{constant treatment effect} assumption)
\item  Let $Y_{i,t}$ be the \textbf{observed} outcome variable for individual $i$ at time $t$
\item  $X_{it}$ be the treatment variable: It is $1$ if individual $i$ is treated and $0$ if otherwise. 
\item Let $Y_{it}(0)$ denote a \textbf{potential} outcome if subject $i$ is not treated at time $t$ and $Y_{it}(1)$ be the same if $i$ is treated. 
\item Then $Y_{it}$ can be split into
\[
\begin{aligned}
Y_{it} & = Y_{it}(1)X_{it}+Y_{it}(0)(1-X_{it})\\
&=Y_{it}(0)+(Y_{it}(1)-Y_{it}(0))X_{it} \\
\end{aligned}
\] 
\end{itemize}
\end{frame}

\begin{frame}
\frametitle{Potential Outcome Framework}
Two takeaways
\begin{itemize}
\item \textbf{Fundamental problem of missing data}: We can only observe at most one of $Y_{it}(1)$ and $Y_{it}(0)$, since individual $i$ cannot be treated and untreated simultaneously at time $t$
\begin{itemize}
\item For us to make statements about the treatment effect, we need to be sure about what the missing outcome looks like. 
\item Perfect randomization, randomization conditional on observables, instrumental variables on treatment variable $X_{it}$...
\end{itemize}
\item \textbf{Average treatment effect}: Average treatment effect is defined as
\small{\[
ATE = E[Y_{it}(1)-Y_{it}(0)]=\frac{1}{N}\sum_{i=1}^N(Y_{it}(1)-Y_{it}(0))
\]}\normalsize
which is obtained from the coefficient of the $X_{it}$ variable of the regression
\[
Y_{it}=\beta_0+\beta_1X_{it}+u_{it}
\]
\end{itemize}
\end{frame}

\begin{frame}
\frametitle{Average Treatment Effect}
Regressional form
\begin{itemize}
\item We will assume that the experiment is \textbf{perfectly randomized} 
\begin{itemize}
\item The treated individuals and controlled individuals are identical except for treatment status, or that $E[u_{it}|X_{it}]=0$ for all possible $X_{it}$ values. 
\end{itemize}
\item Derive the expected value of $Y_{it}$ separately for the treated and controlled individuals. For the controlled ($X_{it}=0$, so that $Y_{it}=Y_{it}(0)$):
\[
E[Y_{it}|X_{it}=0]=E[Y_{it}(0)]=\beta_0
\]
and for the treated, ($X_{it}=1$, so that $Y_{it}=Y_{it}(1)$)
\[
E[Y_{it}|X_{it}=1]=E[Y_{it}(1)]=\beta_0+\beta_1
\]
\end{itemize}
\end{frame}

\begin{frame}
\frametitle{Average Treatment Effect}
Regressional form
\begin{itemize}
\item Then, the average treatment effect can be characterized as
\[
ATE = E[Y_{it}(1)-Y_{it}(0)]=(\beta_0+\beta_1)-\beta_0=\beta_1
\]
where subtraction among $E[Y_{it}|X_{it}=1]$ and $E[Y_{it}|X_{it}=0]$ is possible since we are assuming perfect randomization.
\item Under perfect randomization, identifying the average treatment effect is equivalent to obtaining the $\beta_1$ coefficient through an OLS process.
\item However this is possible in rare circumstances...
\end{itemize}
\end{frame}

\begin{frame}
\frametitle{Average Treatment Effect: No constant treatment effects}
Approach
\begin{itemize}
\item Define $\beta_{1i}= Y_{it}(1)-Y_{it}(0)$ and let $\beta_1=E[\beta_{1i}]$. 
\item The potential outcome framework can be formally written as
\[
\begin{aligned}
Y_i & = Y_i(1)X_i+Y_i(0)(1-X_i)\\
&=Y_i(0)+(Y_i(1)-Y_i(0))X_i \\
&=E[Y_i(0)]+(Y_i(1)-Y_i(0))X_i+Y_i(0)-E[Y_i(0)]\\
&=\beta_0+\beta_{1i}X_i+u_i   \\
\end{aligned} 
\]
\item Furthermore, 
\[
Y_i = \beta_0 + \beta_1X_i+\underbrace{(\beta_{1i}-\beta_1)X_i+u_i}_{=v_i}
\]
\end{itemize}
\end{frame}

\begin{frame}
\frametitle{Average Treatment Effect: No constant treatment effects}
Approach
\begin{itemize}
\item As long as we have perfect randomization, even $E[v_i|X_i]=0$ will hold. 
\[
\begin{aligned}
E[v_i|X_i]&=E[(\beta_{1i}-\beta_1)X_i+u_i|X_i]\\
&=E[(\beta_{1i}-\beta_1)X_i|X_i]+E[u_i|X_i]\\
&=E[(\beta_{1i}-\beta_1)|X_i]X_i+E[u_i|X_i]\\
&=0\\
\end{aligned}
\]
\item With perfect randomization, OLS gives an unbiased estimate of the average treatment effect, whether we assume constant treatment effects or not. 
\end{itemize}
\end{frame}

\begin{frame}
\frametitle{Validity Threats}
Validity threats
\begin{itemize}
\item We cannot automatically subtract $E[Y_{it}|X_{it}=1]$ and $E[Y_{it}|X_{it}=0]$ under \textbf{imperfect randomization}. 
\item If the \textbf{attrition rate is nonrandom} - if it differs by a treatment status - we end up with a biased estimate of the treatment effect. 
\item In terms of experimental procedure, \textbf{failure to comply to experiment protocol} and other \textbf{experimental effects} that rises through peculiar behaviors of the experimental and the experiment subject could also serve as a validity threat. 
\item  Constant treatment effect assumption that we have imposed on ourselves may not be accurate. 
\item There are external validity threats when the sample is not representative, is not replicable in other settings, and the results may have general equilibrium effects.
\end{itemize}
\end{frame}

\begin{frame}
\frametitle{Validity Threats}
Addressing validity threats
\begin{itemize}
\item If the treated and the controlled differ in some observed characteristics, we can simply include control variables $Z_{it}$
\begin{itemize}
\item This gives us ATE conditional on observables and identify treatment effects for people with different treatment effects
\end{itemize}
\item Instrumental variable approach: If there exists a $Z_{it}$ variable that influences the treatment status $X_{it}$ and uncorrelated with $u_{it}$, we can use $Z_{it}$ to instrument $X_{it}$ and run 2SLS regression
\begin{itemize}
\item The predicted value of $X_{it}$: The probability of being treated. 
\item 2SLS estimates the causal effect for those whose value of $X_{it}$ is influenced by $Z_{it}$, putting more weight on those more likely to be treated.
\item This effectively identifies the treatment effect `localized' for those more likely to be treated
\item We call this \textbf{localized average treatment effect} (LATE).
\end{itemize}
\end{itemize}
\end{frame}

\begin{frame}
\frametitle{IV in treatment effects}
Regression
\begin{itemize}
\item Let $Z_{it}$ be an instrument for $X_{it}$ and apply 2SLS method in this manner
\begin{gather*}
X_{it} = \pi_0+\pi_{1i}Z_{it}+e_{it} \ \text{(First stage)}\\
Y_{it} = \beta_0+\beta_{1i}X_{it}+u_{it} \ \text{(Equation of interst)}
\end{gather*}
\item Obtain $\hat{X}_{it}=\hat{\pi_0}+\hat{\pi}_{1i}X_{it}$ - the predicted probability of being treated. 
\item Put $\hat{X}_{it}$ in place of $X_{it}$ in the equation of interest. 
\item If $\beta_{1i}, \pi_{1i}$ are independent of $(u_{it},v_{it}, Z_{it})$, $E(\pi_{1i})\neq0$, then
\[
\hat{\beta}_{1}\xrightarrow{p}\frac{E(\beta_{1i}\pi_{1i})}{E(\pi_{1i})} \ \text{(Refer to appendix 13.2)}
\]
\item Ultimately, $\hat{\beta}_1$ can be written as 
\[
\hat{\beta}_{1}\xrightarrow{p} E[\beta_{1i}]+\frac{cov(\beta_{1i},\pi_{1i})}{E(\pi_{1i})} = \beta_1+\frac{cov(\beta_{1i},\pi_{1i})}{E(\pi_{1i})}
\]
\item Treatment effect is large for individuals for whom the effect of the instrument is large.
\end{itemize}
\end{frame}
%%%%%%%%%%%
\end{document}
